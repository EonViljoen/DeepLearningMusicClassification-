\documentclass[10pt,a4paper]{article}
\usepackage[utf8]{inputenc}
\usepackage{amsmath}
\usepackage{amsfonts}
\usepackage{amssymb}
\begin{document}
	\pagenumbering{gobble}
	\paragraph{Music Classification using Machine Learning}
		\subparagraph{ITRW 367}
		\subparagraph{Eon Viljoen}
		\subparagraph{BSc Hons 2020}
		\subparagraph{Mnr D. Snyman}
		\subparagraph{Computer Science and Information Technology}
		\subparagraph{NWU Potchefstroom Campus}
		
	\newpage
	\tableofcontents
	
	\newpage
	\pagenumbering{arabic}
	\section{Project Description}
		

	\section{Problem description and background}
		Metadata, beingdefined as data about data, is used to extract traits from (?objects).	With the example 		of a song track, metadata gives information about the song name, artist performing it, location 					stored, and the subject of this project, genre.
		
		Music is generally grouped together according to metadata, creating a scenario where if their is none 		present, it can't be grouped. My project aims to lessens this problem through the use of machine 				learning to be able to classifly song tracks according to genre.
		
	\section{Aims and objectives of project}
	\section{Procedures and methods that will be used }
	\section{Approach to project management and project plan }
	\section{Description of development platform, resources, and environments that will be used}
	\section{Ethical and legal implications and dealing with these}
	\section{Provisional chapter division}
	\section{References}
	\section{Research proposal according to prescribed template}
		
\end{document}